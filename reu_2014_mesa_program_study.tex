\documentclass[11pt]{article}
\usepackage[margin=1in]{geometry}
\usepackage{titlesec}
\usepackage{natbib}
\usepackage{setspace}
\usepackage{hyperref}
\usepackage{booktabs} % for nicer tables (optional)
\usepackage{caption}  % for controlling captions (optional)
\captionsetup[table]{skip=5pt} % space between table and caption

\titleformat{\section}{\large\bfseries}{\thesection}{1em}{}
\titleformat{\subsection}{\normalsize\bfseries}{\thesubsection}{1em}{}

\title{Students’ Self-Efficacy, Perception of Engineering, and Engineering Interest in the Context of MESA}
\author{
    Siham Elmali\thanks{First author} \\
    Johns Hopkins University \\
    \texttt{selmali1@jhu.edu}
    \and
    Dr. Presentacion Rivera-Reyes \\
    Utah State University \\
    \texttt{p.rivera@usu.edu}
    \and
    Dr. Oenardi Lawanto \\
    Utah State University \\
    \texttt{olawanto@usu.edu}
    \and
    Deborah Rose Donaldson\footnotemark[1] \\
    California Baptist University \\
    \texttt{DeborahRose.Donaldson@calbaptist.edu}
}

\date{Dec 14, 2015}

\begin{document}

\maketitle
\begin{abstract}
\small
\setstretch{1.1}
\itshape
The focus of this paper is on the self-efficacy of students, perception of engineering and interest in engineering in the context of the MESA program, and how their exposure to MESA activities could influence their career choice. MESA is a nationally recognized program for its effectiveness as an academic program to expose K--12 students to math, engineering, and science-related topics. Two research questions related to the relationship and difference between participants who choose to pursue a math, engineering, or science major compared with participants who choose to pursue another major in the context of MESA activities with self-efficacy, perception about engineering, and interest in engineering were considered. Four hundred and thirty-five students aged K--12 in four states participated in this study. A previously validated survey instrument was applied to the participants. The results showed that there is a relationship between MESA activities and student self-efficacy, perception about engineering, and interest in engineering. Additionally, the results revealed a statistically significant difference in the levels of students’ self-efficacy, perception of engineering, and interest in engineering between the students who decided to enroll in a math, engineering, or science major and those who chose another major. Exposure to MESA activities influences students' self-efficacy, perception of engineering, and interest in engineering, and therefore impacts their decision to pursue an engineering major. Future analysis is recommended to determine the influence of these constructs on MESA program outcomes.
\end{abstract}



\section{Introduction}

MESA stands for Mathematics Engineering Science Achievement and is nationally recognized for its effectiveness as an academic development program that works to engage over 45,000 K--16 students annually who display interest in math, science, and technology-related studies. Most MESA participants come from low-income families, attend or have attended low-performing schools, and are the first in their families to attend college \cite{mesa2014}.

The MESA program began in 1970 with a vision “to support the national science and mathematics educational agenda by ensuring that MESA students develop a high level of literacy in mathematics and science so they can play a leading role within an increasingly technology-based world.” The program is said to have produced such worthwhile results that “it sells itself” \cite{somerton1994}. Because of the success that MESA has achieved, a survey of MESA participants was conducted. The purpose of this paper is to analyze the results of this survey to determine the role MESA has in the following three constructs: students’ self-efficacy, perception of engineering, and interest in engineering. These constructs have the potential to be predictors in the decision to pursue science and mathematics-based careers \cite{hailey2011}.

\subsection{Self-Efficacy}

According to Bandura (1977), self-efficacy is one’s belief in their ability to succeed in a particular situation. In other words, self-efficacy relates to an individual’s confidence that they can accomplish goals, tasks, and challenges in their life. Research has shown that “the lack of progress in recruiting and retaining women and minorities in engineering is partially due to students’ self-efficacy,” or low self-efficacy \cite{jordan2011}. Bandura theorized that psychological tools can be used to alter an individual’s self-efficacy both in level and in strength \cite{bandura1977}.

Further research has examined what tools have a positive impact on students’ self-efficacy. In one study, 118 self-selected students were asked a series of questions to reflect on how the structure of a physics course affected their self-efficacy. Group learning and hands-on activities were reported as positive influences \cite{gaffney2013}.

Another study examined the impact of an engineering mentorship program on African-American high school students. Researchers aimed to determine whether there was a significant difference in the self-efficacy of mentored versus non-mentored students. Although some differences were noted, they were not statistically significant. The results were consistent with Garet et al. (2001), who recommended at least 100 hours of reform activities to achieve an effect \cite{denson2010}.

Similarly, a study at Louisiana State University examined students who attended a 5-day camp, the Biology Intensive Orientation for Students (BIOS), compared with non-attendees. Students in the camp, placed in small groups with undergraduate and graduate mentors, outperformed their peers and exhibited higher self-efficacy in Biology 1201 \cite{wheeler2014}. Thus, hands-on activities and mentorship programs appear to positively influence self-efficacy.

\subsection{Perception About Engineering}

National demographic data conveys one reality very clearly: America is becoming more diverse at an unprecedented rate. In fact, recent U.S. Census Bureau projections indicate that racial and ethnic minorities will make up more than half of the national population by 2050 \cite{zarske2012}. Additionally, by the year 2030, the U.S. is projected to face a labor shortage in Science, Engineering, Technology, and Math (STEM) fields due to declining interest among high school students. Unfortunately, students from minority backgrounds are disproportionately less interested in STEM careers. Currently, these groups represent only 11\% of the STEM workforce \cite{chubin2005}. Thus, the recruitment and retention of minorities in STEM fields remains a national concern.

Many low-income, ethnic minority students describe science and engineering as boring and frustrating disciplines \cite{mark2013}. These negative perceptions may stem from misconceptions about engineering and its careers, potentially leading students to pursue alternate paths. 

Numerous studies have examined K--12 students' perceptions of engineering. A groundbreaking study led by Mead and Metraux in the late 1950s revealed that high school students viewed scientists as lab workers wearing white coats and glasses, surrounded by test tubes \cite{fralick2009}. In a Dutch study, it was found that students’ preconceptions of STEM careers were shaped by external sources such as friends, media, books, or prior classroom experiences in math and science \cite{korpershoek2013}. Fralick et al. (2009) discovered that students viewed engineers as manual workers with limited cognitive responsibilities, often associated with tasks like building and operating machinery. The study also revealed a general lack of understanding about the roles and responsibilities of engineers.

Despite these misconceptions, engineers are often perceived positively as problem solvers who are respected and contribute significantly to society \cite{marshall2007}. However, the persistence of stereotypical and narrow views of engineering—particularly among minority and undereducated groups—suggests a need for greater awareness and education about the profession.

\subsection{Interest in Engineering}

In addition to the connection between early STEM exposure and later career success, research indicates a relationship between minority students’ interest in math and science during K--12 and their eventual success in STEM fields \cite{museus2011}. Further studies have explored factors influencing students’ choice of college major. For example, Adams et al. (1994) found that 59\% of students identified personal interest in the field as the primary factor influencing their choice \cite{adams1994}. In another study involving 852 college students, interest in the field emerged as the most significant influence on major selection \cite{beggs2008}.

While interest plays a critical role in students' decisions, it cannot develop without meaningful exposure to the field. Ewen (2013) emphasized the importance of introducing students to science and technology at a young age to help cultivate interest \cite{ewen2013}. 

Accordingly, in this paper, \textit{Engineering Interest} is defined to include a student’s enthusiasm for reading about engineering issues, engaging in projects involving engineering principles, learning new physics and math material, using engineering to solve real-world problems, and striving to make the world a better place.

\section{Methodology}

Some researchers have claimed that pre-engineering programs have been tremendously successful \cite{lam2005} due to their ability to expose high school students to STEM activities, which in turn helps develop positive perceptions of engineering and related careers. The MESA project included various activities—such as field trips, meetings with professionals, and hands-on projects—that aimed to improve students’ self-efficacy, broaden their understanding of the societal impact of engineering, and help them envision themselves in STEM careers.

The purpose of this paper is to analyze the results of the MESA survey in relation to three constructs: students’ self-efficacy, perception of engineering, and interest in engineering. Two research questions guided this study:

\begin{enumerate}
    \item Is there a relationship between MESA activities and students’ self-efficacy, perception about engineering, and interest in engineering?
    \item How do students’ self-efficacy, perception about engineering, and interest in engineering differ between those who choose a MESA-related major (math, engineering, or science) and those who choose another major?
\end{enumerate}

\subsection{Participants}

The study was conducted among 711 high school students from MESA-participating schools across four U.S. states: California, Utah, Maryland, and Washington. 

\begin{table}[h]
\centering
\begin{tabular}{lr}
\hline
\textbf{State} & \textbf{Number of Students} \\
\hline
California & 412 \\
Utah       & 175 \\
Maryland   & 87 \\
Washington & 37 \\
\hline
\textbf{Total} & 711 \\
\hline
\end{tabular}
\caption{Sample Breakdown Based on State}
\label{tab:state-breakdown}
\end{table}

Out of the 711 surveys distributed, 276 (38.8\%) were omitted due to incompletion, resulting in 435 valid responses used in the final analysis. 

\begin{table}[h]
\centering
\begin{tabular}{lr}
\hline
\textbf{Grade Level} & \textbf{Percentage} \\
\hline
9th Grade  & 25.0\% \\
10th Grade & 14.5\% \\
11th Grade & 23.8\% \\
12th Grade & 26.4\% \\
\hline
\end{tabular}
\caption{Sample Breakdown Based on Year in High School}
\label{tab:grade-breakdown}
\end{table}

\subsection{Survey Instrument}

The survey instrument was initially pilot tested in 2007 to ensure clarity and participant comprehension. In 2011, it was further refined to specifically measure high school students’ self-efficacy, perception of engineering, and interest in engineering \cite{hailey2011}. A factor analysis using the maximum likelihood method identified three subscales—self-efficacy, perception, and interest—based on a factor loading cutoff of 0.44, following Sedlmeier and Gigerenzer’s 24-year power analysis recommendations \cite{sedlmeier1989}. These findings support the reliability and validity of the instrument.

In 2012, a focus group was conducted to explore the unique aspects of MESA for underrepresented students. Based on this focus group, additional questions related to the impact of MESA were added to the survey \cite{denson2012}. A second pilot study was conducted in 2013, followed by a content validity assessment by a five-member advisory board. Based on their feedback, the finalized survey instrument was administered to students in California, Utah, Washington, and Maryland \cite{hailey2013}.

The full instrument includes 35 questions, primarily formatted using a 5-point Likert scale ranging from ``Strongly Disagree'' to ``Strongly Agree.'' The survey sections relevant to this research are as follows:

\begin{itemize}
    \item \textbf{MESA Activity Involvement:} Sections 18, 19, and 21 (17 items total)
    \item \textbf{Self-Efficacy:} Sections 5 (6 items) and 6 (5 items)
    \item \textbf{Perception About Engineering:} Sections 7 (6 items) and 8 (6 items)
    \item \textbf{Engineering Interest:} Sections 9 (8 items) and 10 (6 items)
    \item \textbf{Demographics:} Section 25 (gender, ethnicity, grade level)
\end{itemize}

\subsection{Data Collection}

Data was collected on a voluntary basis. Upon agreement to participate, students in California, Utah, and Maryland completed an online anonymous survey about their involvement in the MESA program during the 2012--2013 academic year. In contrast, participants in Washington completed printed survey packets. Utah participants took the survey in the spring of 2013, while participants from California, Maryland, and Washington completed it in the fall of the same year. The survey, administered during class hours under instructor supervision, took approximately 30--45 minutes to complete. Students were informed of their right to withdraw at any point without consequence.

\subsection{Data Analysis}

This quantitative study used IBM SPSS Statistics to analyze the data and address the research questions. A Spearman correlation analysis was conducted to explore the relationship between MESA activities and students’ self-efficacy, addressing the first research question. For the second research question, descriptive statistics were used to examine how minority students' participation in MESA correlated with their perceptions of engineering. A regression analysis was used to examine causal relationships among four constructs: self-efficacy, perception of engineering, MESA activities, and engineering interest, as they relate to students’ career choices.

Survey responses were scored using a 5-point Likert scale: “Strongly Disagree” (1), “Disagree” (2), “Neutral” (3), “Agree” (4), and “Strongly Agree” (5). Negatively worded items, such as “engineers have boring desk jobs,” were reverse-coded during analysis—i.e., a response of “Strongly Agree” was scored as 1 instead of 5.

\section{Results}

\subsection{Addressing Research Question 1: Is there a relationship between MESA activities and students’ self-efficacy, perception of engineering, and interest in engineering?}

To address this question, composite scores were created for each construct.

\begin{itemize}
    \item \textbf{MESA Activities:} Derived from Sections 18, 19, and 21 (17 items). Five of these items were originally on the 4-point Likert scale and were converted to 5-point for consistency.
    \item \textbf{Self-Efficacy:} Derived from Sections 5 and 6 (11 items total).
    \item \textbf{Perception of Engineering:} Derived from sections 7 and 8 (12 items total).
    \item \textbf{Interest in Engineering:} Derived from Sections 9 and 10 (14 items total).
\end{itemize}

Before performing further statistical analyzes, the normality of the data was evaluated using skewness, kurtosis, histograms, and QQ plots. Table~\ref{tab:normality} summarizes the results of the normality analysis.

\begin{table}[h]
\centering
\small
\begin{tabular}{p{3.5cm} c c p{4.5cm}}
\hline
\textbf{Variable} & \textbf{Skewness} & \textbf{Kurtosis} & \textbf{Interpretation} \\
                  & \textbf{(SE = 0.12)} & \textbf{(SE = 0.23)} &  \\
\hline
MESA Activities           & -0.49 & 0.79  & Approximately Normal \\
Self-Efficacy             & -0.29 & 0.08  & Approximately Normal \\
Perception of Engineering & -0.49 & 4.80  & Acceptable (visual confirmation) \\
Interest in Engineering   & -0.39 & 0.22  & Approximately Normal \\
\hline
\end{tabular}
\caption{Normality Assessment of Composite Scores}
\label{tab:normality}
\end{table}


Visual inspections using histograms and Q-Q plots supported the interpretation that the data approximated the normal distribution, justifying the use of parametric statistical methods \cite{bulmer1979}.

\begin{table}[h]
\centering
\caption{Skewness and Kurtosis for MESA Activities, Self-Efficacy, Perception About Engineering, and Interest in Engineering}
\label{tab:skew_kurt}
\begin{tabular}{lccccc}
\hline
\textbf{Variable} & \textbf{N} & \textbf{Skewness} & \textbf{SE Skew} & \textbf{Kurtosis} & \textbf{SE Kurtosis} \\
\hline
MESA Activities & 435 & -0.49 & 0.12 & 0.79 & 0.23 \\
Self-Efficacy & 435 & -0.29 & 0.12 & 0.08 & 0.23 \\
Perception About Engineering & 435 & 0.49 & 0.12 & 4.79 & 0.23 \\
Interest in Engineering & 435 & -0.39 & 0.12 & 0.22 & 0.23 \\
\hline
\end{tabular}
\begin{flushleft}
\textit{Note.} a Skewness values close to 0 and kurtosis values close to 3 indicate a normal distribution.
\end{flushleft}
\end{table}
An analysis using SPSS was conducted to compute Pearson correlation coefficients between MESA activities and students’ self-efficacy, perception about engineering, and interest in engineering. Results are presented in Table~\ref{tab:correlations}. A positive correlation was observed between:

\begin{itemize}
    \item MESA activities and students’ self-efficacy, $r(435) = .44$, $p < .01$
    \item MESA activities and perception about engineering, $r(435) = .39$, $p < .01$
    \item MESA activities and interest in engineering, $r(435) = .55$, $p < .01$
\end{itemize}

These results suggest that students who engaged more with MESA activities were likely to report higher self-efficacy, more positive perceptions of engineering, and stronger interest in engineering careers.

\begin{table}[h]
\centering
\begin{tabular}{lcc}
\hline
\textbf{Variables} & \textbf{N} & \textbf{Pearson Correlation} \\
\hline
MESA Activities – Self-Efficacy & 435 & .44\textsuperscript{**} \\
MESA Activities – Perception About Engineering & 435 & .39\textsuperscript{**} \\
MESA Activities – Interest in Engineering & 435 & .55\textsuperscript{**} \\
\hline
\end{tabular}
\caption{Relationship Between MESA Activities and Student Constructs}
\label{tab:correlations}
\begin{flushleft}
\textit{Note.} \textsuperscript{**} Correlation is significant at the 0.01 level (2-tailed).
\end{flushleft}
\end{table}

\subsection{\textbf{Addressing Research Question 2}: Differences Based on the intended major}


To address the second research question, Section 41 of the MESA instrument was used. This section asked students to identify the academic fields they intended to pursue by selecting from the following list: math, science, engineering, or none of the above. Participants who selected math, science, or engineering (or any combination thereof) were assigned a value of “1” (MESA major group). Those who selected “none of the above” were assigned a value of “0” (non-MESA major group). Participants who left this section blank were excluded from the analysis.

Descriptive statistics for both groups are shown in Table~\ref{tab:major_diff}. Students in the MESA major group (N = 295) had higher average scores across all three constructs (i.e. self-efficacy, perception of engineering, and interest in engineering) compared to those in the non-MESA major group (N = 95).


\begin{table}[h]
\centering
\begin{tabular}{lcccc}
\hline
\textbf{Construct} & \textbf{Group} & \textbf{N} & \textbf{Mean} & \textbf{SD} \\
\hline
Self-Efficacy & MESA Major & 295 & 3.80 & 0.76 \\
              & Other Major & 95 & 3.31 & 0.61 \\
\hline
Perception About Engineering & MESA Major & 295 & 3.61 & 0.45 \\
                             & Other Major & 95 & 3.44 & 0.50 \\
\hline
Interest in Engineering & MESA Major & 295 & 3.73 & 0.68 \\
                         & Other Major & 95 & 3.01 & 0.71 \\
\hline
\end{tabular}
\caption{Descriptive Statistics by Intended Major}
\label{tab:major_diff}
\end{table}

Overall, the data indicate that students intending to pursue a MESA major demonstrate stronger self-efficacy, more favorable perceptions of engineering, and greater interest in the field compared to their peers selecting other majors.

\begin{table}[h]
\centering
\begin{tabular}{lcccc}
\hline
\textbf{Variable} & \multicolumn{2}{c}{\textbf{MESA Major (N = 295)}} & \multicolumn{2}{c}{\textbf{Other Major (N = 95)}} \\
 & \textbf{M} & \textbf{SD} & \textbf{M} & \textbf{SD} \\
\hline
Self-Efficacy & 3.80\textsuperscript{a} & 0.76 & 3.31\textsuperscript{a} & 0.61 \\
Perception About Engineering & 3.61\textsuperscript{a} & 0.45 & 3.44\textsuperscript{a} & 0.50 \\
Interest in Engineering & 3.73\textsuperscript{a} & 0.68 & 3.01\textsuperscript{a} & 0.71 \\
\hline
\end{tabular}
\caption{Descriptive Statistics for Participants Who Chose MESA Major and Those Who Chose Another Major}
\label{tab:descriptive_major}
\begin{flushleft}
\textit{Note.} \textsuperscript{a} Scores are based on a 5-point Likert scale: (1) Strongly disagree, (2) Disagree, (3) Neither agree nor disagree, (4) Agree, (5) Strongly agree.
\end{flushleft}
\end{table}
An independent samples \textit{t}-test was conducted to determine whether significant differences existed in self-efficacy, perception about engineering, and interest in engineering between students who chose a MESA major and those who chose another major. Results are presented in Table~\ref{tab:ttest_major}. Significant differences were observed for all three constructs.

\begin{table}[h]
\centering
\begin{tabular}{lccc}
\hline
\textbf{Variable} & \textbf{\textit{t}} & \textbf{df} & \textbf{Sig. (2-tailed)} \\
\hline
Self-Efficacy & 5.70 & 388 & .000\textsuperscript{*} \\
Perception About Engineering & 3.15 & 388 & .002\textsuperscript{*} \\
Interest in Engineering & 8.86 & 388 & .000\textsuperscript{*} \\
\hline
\end{tabular}
\caption{Independent Samples \textit{t}-Test for Students Choosing MESA vs. Other Majors}
\label{tab:ttest_major}
\begin{flushleft}
\textit{Note.} \textsuperscript{*} $p < .05$
\end{flushleft}
\end{table}

\section{Conclusions}

The Pearson correlation tests revealed significant positive relationships between MESA activities and students’ self-efficacy, perception of engineering, and interest in engineering. According to Cohen’s (1988) guidelines, these correlations ranged from moderate to strong.

Students involved in MESA activities demonstrated increased self-efficacy, consistent with prior studies showing that hands-on activities and mentorship positively influence students’ confidence \citep{denson2010, gaffney2013, wheeler2014}. This finding aligns with the significant differences observed between participants who chose a MESA major and those who chose a non-MESA major.

Similarly, MESA participants exhibited more favorable perceptions of engineering. This contrasts with research indicating that K–12 students often hold limited or stereotypical views about engineering careers \citep{fralick2009, korpershoek2013, marshall2007}. Exposure to MESA activities appears to positively shift these perceptions, potentially influencing students’ academic and career choices.

Moreover, involvement in MESA activities was associated with greater interest in engineering. Prior research emphasizes that interest develops through exposure and early engagement with science and technology \citep{beggs2008, ewen2013}. MESA’s hands-on projects and engineering-focused experiences help foster this interest, as reflected in the significant differences observed between students who chose MESA majors and those who did not.

This study used correlational methods to explore the relationships between MESA participation and key constructs. Future research should employ multivariate analyses to examine how these constructs interact and influence students’ decisions to pursue MESA majors, thereby further informing the program’s goals and impact.

\section*{Acknowledgment}

This material is based upon work supported by the National Science Foundation under Grant No. 1262806. Any opinions, findings, conclusions, or recommendations expressed in this material are those of the author(s) and do not necessarily reflect the views of the National Science Foundation.

\vspace{1em}

\noindent\textbf{Advisory Board Members}
\begin{itemize}
  \item \textbf{Frances Lawrenz} – Advisory Board Chair and Associate Vice President for Research and Professor of Educational Psychology, University of Minnesota.
  \item \textbf{Vincent Childress} – Original Board Member and Professor, School of Technology, North Carolina A\&T State University.
  \item \textbf{Eve Riskin} – Original Board Member, Associate Dean and Professor, College of Engineering, University of Washington.
  \item \textbf{Anne Hunt} – New Board Member (replacing Dicky Ng), Director of the Office of Methodological Data Science, Emma Eccles Jones College of Education and Human Services, Utah State University.
  \item \textbf{Patricia Richardson} – New Board Member, former MESA Advisor in the Los Angeles School District.
\end{itemize}

\bibliographystyle{plain}
\begin{thebibliography}{99}

\bibitem{adams1994}
Adams, S. H., Pryor, L. J., \& Adams, S. L. (1994). Attraction and retention of high-aptitude students in accounting: An exploratory longitudinal study. \textit{Issues in Accounting Education, 9}(1), 45–58.

\bibitem{bandura1977}
Bandura, A. (1977). \textit{Self-efficacy: Toward a unifying theory of behavioral change}. Psychological Review, 84(2), 191–215.

\bibitem{beggs2008}
Beggs, J. M., Bantham, J. H., \& Taylor, S. (2008). Distinguishing the factors influencing college students’ choice of a major. \textit{College Student Journal, 42}, 381–394.

\bibitem{cohen1988}
Cohen, J. (1988). \textit{Statistical power analysis for the behavioral sciences} (2nd ed.). Hillsdale, NJ: Erlbaum.

\bibitem{chubin2005}
Chubin, D. E., May, G. S., \& Babco, E. L. (2005). Diversifying the engineering workforce. \textit{Journal of Engineering Education, 94}, 73–86.

\bibitem{denson2012}
Denson, C., Austin, C. Y., \& Hailey, C. E. (2012). Investigating unique aspects of the MESA program for underrepresented students. In \textit{American Society for Engineering Education}.

\bibitem{denson2010}
Denson, C. D., \& Hill, R. B. (2010). Impact of an engineering mentorship program on African-American male high school students’ perceptions and self-efficacy.

\bibitem{ewen2013}
Ewen, B. (2013). How interests in science and technology have taken women to an engineering career. \textit{Asia-Pacific Forum on Science Learning \& Teaching, 14}(1), 1–22.

\bibitem{fralick2009}
Fralick, B., Kearn, J., Thompson, S., \& Lyons, J. (2009). How middle schoolers draw engineers and scientists. \textit{Journal of Science Education \& Technology, 18}(1), 60–73.

\bibitem{gaffney2013}
Gaffney, J. D., Gaffney, A. L. H., Usher, E. L., \& Mamaril, N. A. (2013). How an active-learning class influences physics self-efficacy in pre-service teachers. In \textit{2012 Physics Education Research Conference} (Vol. 1513, No. 1, pp. 134–137). AIP Publishing.

\bibitem{hailey2013}
Hailey, C., Austin, C., \& Denson, C. (2013). Preview of Award 1020019—Annual Project Report.

\bibitem{hailey2011}
Hailey, C., Austin, C., Denson, C., \& Householder, D. (2011). Investigating influences of the MESA program upon underrepresented students. Paper presented at the meeting of the American Society for Engineering Education, San Antonio, TX.

\bibitem{jordan2011}
Jordan, K. L., Amato-Henderson, S., Sorby, S. A., \& Haut Donahue, T. L. (2011). Are there differences in engineering self-efficacy between minority and majority students across academic levels? In \textit{American Society for Engineering Education}.

\bibitem{korpershoek2013}
Korpershoek, H., Kuyper, H., Bosker, R., \& van der Werf, G. (2013). Students’ preconceptions and perceptions of science-oriented studies. \textit{International Journal of Science Education, 35}(14), 2356–2375.

\bibitem{lam2005}
Lam, P., Srivatsan, T., Doverspike, D., Vesalo, J., \& Mawasha, P. (2005). A ten year assessment of the pre-engineering program for under-represented, low income and/or first generation college students at the University of Akron. \textit{Journal of STEM Education: Innovations \& Research, 6}(3/4), 14–20.

\bibitem{mark2013}
Mark, S., DeBay, D., Lin Zhang, L., Haley, J., Patchen, A., Wong, C., \& Barnett, M. (2013). Coupling social justice and out of school time learning to provide opportunities to motivate, engage, and interest underrepresented populations in STEM fields. \textit{Career Planning \& Adult Development Journal, 29}(2), 93–105.

\bibitem{marshall2007}
Marshall, H., McClymont, L., \& Joyce, L. (2007). Public attitudes to and perceptions of engineering and engineers 2007. The Royal Academy of Engineering and the Engineering and Technology Board, 64. Retrieved July 23, 2014, from \url{http://www.raeng.org.uk}

\bibitem{museus2011}
Museus, S. D., \& Palmer, R. T. (2011). \textit{Racial and ethnic minority students' success in STEM education}. San Francisco, Calif.: Jossey-Bass Inc.

\bibitem{sedlmeir1989}
Sedlmeir, P., \& Gigerenzer, G. (1989). Do studies of statistical power have an effect on the power of studies? \textit{Psychological Bulletin, 105}(2), 309–316.

\bibitem{somerton1994}
Somerton, W. H., Smith, M., Finnell, R., \& Fuller, T. (1994). \textit{The MESA way: A success story of nurturing minorities for math/science based careers}. San Francisco, Calif.: Caddo Gap Press.

\bibitem{mesa2014}
What’s New in MESA | MESA USA. (n.d.). Retrieved July 21, 2014, from \url{http://mesausa.org/}

\bibitem{wheeler2014}
Wheeler, E. R., \& Wischusen, S. M. (2014). Development self-regulation and self-efficacy: A cognitive mechanism for success of biology boot camps. \textit{Electronic Journal of Science Education, 18}(1).

\bibitem{zarske2012}
Zarske, M., Yowell, J., Ringer, H., Sullivan, J., \& Quinonez, P. (2012). The Skyline TEAMS model: A longitudinal look at the impacts of K-12 engineering on perception, preparation and persistence. \textit{Advances in Engineering Education, 3}(2), 1–25.

\end{thebibliography}


\end{document}
